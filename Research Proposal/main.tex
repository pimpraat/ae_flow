\documentclass{article}
\usepackage[utf8]{inputenc}

\title{Add your title }
\author{Names/Surnames of the group members }
\date{April 2023}

\begin{document}

\maketitle

\section*{Deep learning 2-course research proposal}


When preparing the research proposal for your mini-project you should focus mainly on two parts. Firstly, on providing the research plan that you intend to follow during the development of your mini-project, as well as, the distribution of the workload among the members of your group. The research plan needs to be submitted at the end of the third week (April the 23rd). Note that during the grading process, we will make use of this document as a reference. Finally, the length of the proposal should not be more than 2 pages.


\section{Executive Summary}

(\textbf{1-2 paragraphs}): A summary of the specific research problem that you intend to work on for the DL2 course. This summary can be based on the provided papers that we will provide to you (in the description of the mini-project) or based on a literature review on the specific domain and the state-of-the-art research. Include Research objectives, Significance, and a brief description of the proposed Work. Finally, a comment on the anticipated outcomes.

\section{Survey of Background Literature}

\subsection{Background}
(\textbf{1 paragraph}): Give a thorough introduction to the background work relevant to the general problem of the proposed research. Explain the relevant work and standard techniques used in the area, along with any acronyms and jargon typical to the field. Report the progress that has been made in the field and all the experiments that support this progress. What is the current status of this specific domain/problem?

\subsection{Relevance/Impact}
(\textbf{1 paragraph}): In the context provided by the background work, layout the argument that justifies the need for research in this area. Point out the significance of this proposed research to the field. Comment on its potential impact on the development of the scientific area and the society/sector as a whole.

\section{Proposed Methodology}
(\textbf{1-2 paragraphs}): Outline the methodology and the experiments for the proposed research. Justify the choice of methods and experiments as opposed to alternatives. Avoid giving too much detailed information; just outline the general approach and why you chose to use them.

\section{Research Plan}

\subsection{Plan}
(\textbf{1-2 paragraphs}): State the long - and short -term objectives of your research program. Outline specific projects planned to meet these goals, including the timelines for completion of each stage. What are you planning to do for each week of the project?

\subsection{Who does what?}
(\textbf{1-2 paragraphs}):

Add also a distribution of the workload among the group members for all the tasks of your project. You will need to add the distribution of the workload (for the experiments-code/literature review/writing of the tutorial etc).

\section{Resources}
Provide details on the instrumentation and materials needed, along with the estimates of human resources required for each project/activity throughout the lifetime of the project

\subsection{Code of honor statement}
(\textbf{1 paragraph}): A statement with which you agree to follow this plan the proposed timeline and distribution of the workload. You should also add all your signatures and the date.
\\[12pt]


\vfill
\newcommand{\namesigdate}[2][4cm]{%
\begin{minipage}{#1}
    #2 \vspace{1cm}\hrule\smallskip
\end{minipage}
}

\noindent \namesigdate{Ms. Laura Palmer} \ \hfill \namesigdate[4cm]{Mr. Dale Cooper} 
\\
\\
\vspace{1cm}\smallskip \\ \\ 
\noindent \namesigdate{Mr.Tommy Hill} \ \hfill \namesigdate[4cm]{Ms. Audrey Horne}
\\
\\
Date 29/03/2022
\\
\end{document}
